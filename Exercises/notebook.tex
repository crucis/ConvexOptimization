
% Default to the notebook output style

    


% Inherit from the specified cell style.




    
\documentclass[11pt]{article}

    
    
    \usepackage[T1]{fontenc}
    % Nicer default font (+ math font) than Computer Modern for most use cases
    \usepackage{mathpazo}

    % Basic figure setup, for now with no caption control since it's done
    % automatically by Pandoc (which extracts ![](path) syntax from Markdown).
    \usepackage{graphicx}
    % We will generate all images so they have a width \maxwidth. This means
    % that they will get their normal width if they fit onto the page, but
    % are scaled down if they would overflow the margins.
    \makeatletter
    \def\maxwidth{\ifdim\Gin@nat@width>\linewidth\linewidth
    \else\Gin@nat@width\fi}
    \makeatother
    \let\Oldincludegraphics\includegraphics
    % Set max figure width to be 80% of text width, for now hardcoded.
    \renewcommand{\includegraphics}[1]{\Oldincludegraphics[width=.8\maxwidth]{#1}}
    % Ensure that by default, figures have no caption (until we provide a
    % proper Figure object with a Caption API and a way to capture that
    % in the conversion process - todo).
    \usepackage{caption}
    \DeclareCaptionLabelFormat{nolabel}{}
    \captionsetup{labelformat=nolabel}

    \usepackage{adjustbox} % Used to constrain images to a maximum size 
    \usepackage{xcolor} % Allow colors to be defined
    \usepackage{enumerate} % Needed for markdown enumerations to work
    \usepackage{geometry} % Used to adjust the document margins
    \usepackage{amsmath} % Equations
    \usepackage{amssymb} % Equations
    \usepackage{textcomp} % defines textquotesingle
    % Hack from http://tex.stackexchange.com/a/47451/13684:
    \AtBeginDocument{%
        \def\PYZsq{\textquotesingle}% Upright quotes in Pygmentized code
    }
    \usepackage{upquote} % Upright quotes for verbatim code
    \usepackage{eurosym} % defines \euro
    \usepackage[mathletters]{ucs} % Extended unicode (utf-8) support
    \usepackage[utf8x]{inputenc} % Allow utf-8 characters in the tex document
    \usepackage{fancyvrb} % verbatim replacement that allows latex
    \usepackage{grffile} % extends the file name processing of package graphics 
                         % to support a larger range 
    % The hyperref package gives us a pdf with properly built
    % internal navigation ('pdf bookmarks' for the table of contents,
    % internal cross-reference links, web links for URLs, etc.)
    \usepackage{hyperref}
    \usepackage{longtable} % longtable support required by pandoc >1.10
    \usepackage{booktabs}  % table support for pandoc > 1.12.2
    \usepackage[inline]{enumitem} % IRkernel/repr support (it uses the enumerate* environment)
    \usepackage[normalem]{ulem} % ulem is needed to support strikethroughs (\sout)
                                % normalem makes italics be italics, not underlines
    

    
    
    % Colors for the hyperref package
    \definecolor{urlcolor}{rgb}{0,.145,.698}
    \definecolor{linkcolor}{rgb}{.71,0.21,0.01}
    \definecolor{citecolor}{rgb}{.12,.54,.11}

    % ANSI colors
    \definecolor{ansi-black}{HTML}{3E424D}
    \definecolor{ansi-black-intense}{HTML}{282C36}
    \definecolor{ansi-red}{HTML}{E75C58}
    \definecolor{ansi-red-intense}{HTML}{B22B31}
    \definecolor{ansi-green}{HTML}{00A250}
    \definecolor{ansi-green-intense}{HTML}{007427}
    \definecolor{ansi-yellow}{HTML}{DDB62B}
    \definecolor{ansi-yellow-intense}{HTML}{B27D12}
    \definecolor{ansi-blue}{HTML}{208FFB}
    \definecolor{ansi-blue-intense}{HTML}{0065CA}
    \definecolor{ansi-magenta}{HTML}{D160C4}
    \definecolor{ansi-magenta-intense}{HTML}{A03196}
    \definecolor{ansi-cyan}{HTML}{60C6C8}
    \definecolor{ansi-cyan-intense}{HTML}{258F8F}
    \definecolor{ansi-white}{HTML}{C5C1B4}
    \definecolor{ansi-white-intense}{HTML}{A1A6B2}

    % commands and environments needed by pandoc snippets
    % extracted from the output of `pandoc -s`
    \providecommand{\tightlist}{%
      \setlength{\itemsep}{0pt}\setlength{\parskip}{0pt}}
    \DefineVerbatimEnvironment{Highlighting}{Verbatim}{commandchars=\\\{\}}
    % Add ',fontsize=\small' for more characters per line
    \newenvironment{Shaded}{}{}
    \newcommand{\KeywordTok}[1]{\textcolor[rgb]{0.00,0.44,0.13}{\textbf{{#1}}}}
    \newcommand{\DataTypeTok}[1]{\textcolor[rgb]{0.56,0.13,0.00}{{#1}}}
    \newcommand{\DecValTok}[1]{\textcolor[rgb]{0.25,0.63,0.44}{{#1}}}
    \newcommand{\BaseNTok}[1]{\textcolor[rgb]{0.25,0.63,0.44}{{#1}}}
    \newcommand{\FloatTok}[1]{\textcolor[rgb]{0.25,0.63,0.44}{{#1}}}
    \newcommand{\CharTok}[1]{\textcolor[rgb]{0.25,0.44,0.63}{{#1}}}
    \newcommand{\StringTok}[1]{\textcolor[rgb]{0.25,0.44,0.63}{{#1}}}
    \newcommand{\CommentTok}[1]{\textcolor[rgb]{0.38,0.63,0.69}{\textit{{#1}}}}
    \newcommand{\OtherTok}[1]{\textcolor[rgb]{0.00,0.44,0.13}{{#1}}}
    \newcommand{\AlertTok}[1]{\textcolor[rgb]{1.00,0.00,0.00}{\textbf{{#1}}}}
    \newcommand{\FunctionTok}[1]{\textcolor[rgb]{0.02,0.16,0.49}{{#1}}}
    \newcommand{\RegionMarkerTok}[1]{{#1}}
    \newcommand{\ErrorTok}[1]{\textcolor[rgb]{1.00,0.00,0.00}{\textbf{{#1}}}}
    \newcommand{\NormalTok}[1]{{#1}}
    
    % Additional commands for more recent versions of Pandoc
    \newcommand{\ConstantTok}[1]{\textcolor[rgb]{0.53,0.00,0.00}{{#1}}}
    \newcommand{\SpecialCharTok}[1]{\textcolor[rgb]{0.25,0.44,0.63}{{#1}}}
    \newcommand{\VerbatimStringTok}[1]{\textcolor[rgb]{0.25,0.44,0.63}{{#1}}}
    \newcommand{\SpecialStringTok}[1]{\textcolor[rgb]{0.73,0.40,0.53}{{#1}}}
    \newcommand{\ImportTok}[1]{{#1}}
    \newcommand{\DocumentationTok}[1]{\textcolor[rgb]{0.73,0.13,0.13}{\textit{{#1}}}}
    \newcommand{\AnnotationTok}[1]{\textcolor[rgb]{0.38,0.63,0.69}{\textbf{\textit{{#1}}}}}
    \newcommand{\CommentVarTok}[1]{\textcolor[rgb]{0.38,0.63,0.69}{\textbf{\textit{{#1}}}}}
    \newcommand{\VariableTok}[1]{\textcolor[rgb]{0.10,0.09,0.49}{{#1}}}
    \newcommand{\ControlFlowTok}[1]{\textcolor[rgb]{0.00,0.44,0.13}{\textbf{{#1}}}}
    \newcommand{\OperatorTok}[1]{\textcolor[rgb]{0.40,0.40,0.40}{{#1}}}
    \newcommand{\BuiltInTok}[1]{{#1}}
    \newcommand{\ExtensionTok}[1]{{#1}}
    \newcommand{\PreprocessorTok}[1]{\textcolor[rgb]{0.74,0.48,0.00}{{#1}}}
    \newcommand{\AttributeTok}[1]{\textcolor[rgb]{0.49,0.56,0.16}{{#1}}}
    \newcommand{\InformationTok}[1]{\textcolor[rgb]{0.38,0.63,0.69}{\textbf{\textit{{#1}}}}}
    \newcommand{\WarningTok}[1]{\textcolor[rgb]{0.38,0.63,0.69}{\textbf{\textit{{#1}}}}}
    
    
    % Define a nice break command that doesn't care if a line doesn't already
    % exist.
    \def\br{\hspace*{\fill} \\* }
    % Math Jax compatability definitions
    \def\gt{>}
    \def\lt{<}
    % Document parameters
    \title{ExerciciosLista1}
    
    
    

    % Pygments definitions
    
\makeatletter
\def\PY@reset{\let\PY@it=\relax \let\PY@bf=\relax%
    \let\PY@ul=\relax \let\PY@tc=\relax%
    \let\PY@bc=\relax \let\PY@ff=\relax}
\def\PY@tok#1{\csname PY@tok@#1\endcsname}
\def\PY@toks#1+{\ifx\relax#1\empty\else%
    \PY@tok{#1}\expandafter\PY@toks\fi}
\def\PY@do#1{\PY@bc{\PY@tc{\PY@ul{%
    \PY@it{\PY@bf{\PY@ff{#1}}}}}}}
\def\PY#1#2{\PY@reset\PY@toks#1+\relax+\PY@do{#2}}

\expandafter\def\csname PY@tok@w\endcsname{\def\PY@tc##1{\textcolor[rgb]{0.73,0.73,0.73}{##1}}}
\expandafter\def\csname PY@tok@c\endcsname{\let\PY@it=\textit\def\PY@tc##1{\textcolor[rgb]{0.25,0.50,0.50}{##1}}}
\expandafter\def\csname PY@tok@cp\endcsname{\def\PY@tc##1{\textcolor[rgb]{0.74,0.48,0.00}{##1}}}
\expandafter\def\csname PY@tok@k\endcsname{\let\PY@bf=\textbf\def\PY@tc##1{\textcolor[rgb]{0.00,0.50,0.00}{##1}}}
\expandafter\def\csname PY@tok@kp\endcsname{\def\PY@tc##1{\textcolor[rgb]{0.00,0.50,0.00}{##1}}}
\expandafter\def\csname PY@tok@kt\endcsname{\def\PY@tc##1{\textcolor[rgb]{0.69,0.00,0.25}{##1}}}
\expandafter\def\csname PY@tok@o\endcsname{\def\PY@tc##1{\textcolor[rgb]{0.40,0.40,0.40}{##1}}}
\expandafter\def\csname PY@tok@ow\endcsname{\let\PY@bf=\textbf\def\PY@tc##1{\textcolor[rgb]{0.67,0.13,1.00}{##1}}}
\expandafter\def\csname PY@tok@nb\endcsname{\def\PY@tc##1{\textcolor[rgb]{0.00,0.50,0.00}{##1}}}
\expandafter\def\csname PY@tok@nf\endcsname{\def\PY@tc##1{\textcolor[rgb]{0.00,0.00,1.00}{##1}}}
\expandafter\def\csname PY@tok@nc\endcsname{\let\PY@bf=\textbf\def\PY@tc##1{\textcolor[rgb]{0.00,0.00,1.00}{##1}}}
\expandafter\def\csname PY@tok@nn\endcsname{\let\PY@bf=\textbf\def\PY@tc##1{\textcolor[rgb]{0.00,0.00,1.00}{##1}}}
\expandafter\def\csname PY@tok@ne\endcsname{\let\PY@bf=\textbf\def\PY@tc##1{\textcolor[rgb]{0.82,0.25,0.23}{##1}}}
\expandafter\def\csname PY@tok@nv\endcsname{\def\PY@tc##1{\textcolor[rgb]{0.10,0.09,0.49}{##1}}}
\expandafter\def\csname PY@tok@no\endcsname{\def\PY@tc##1{\textcolor[rgb]{0.53,0.00,0.00}{##1}}}
\expandafter\def\csname PY@tok@nl\endcsname{\def\PY@tc##1{\textcolor[rgb]{0.63,0.63,0.00}{##1}}}
\expandafter\def\csname PY@tok@ni\endcsname{\let\PY@bf=\textbf\def\PY@tc##1{\textcolor[rgb]{0.60,0.60,0.60}{##1}}}
\expandafter\def\csname PY@tok@na\endcsname{\def\PY@tc##1{\textcolor[rgb]{0.49,0.56,0.16}{##1}}}
\expandafter\def\csname PY@tok@nt\endcsname{\let\PY@bf=\textbf\def\PY@tc##1{\textcolor[rgb]{0.00,0.50,0.00}{##1}}}
\expandafter\def\csname PY@tok@nd\endcsname{\def\PY@tc##1{\textcolor[rgb]{0.67,0.13,1.00}{##1}}}
\expandafter\def\csname PY@tok@s\endcsname{\def\PY@tc##1{\textcolor[rgb]{0.73,0.13,0.13}{##1}}}
\expandafter\def\csname PY@tok@sd\endcsname{\let\PY@it=\textit\def\PY@tc##1{\textcolor[rgb]{0.73,0.13,0.13}{##1}}}
\expandafter\def\csname PY@tok@si\endcsname{\let\PY@bf=\textbf\def\PY@tc##1{\textcolor[rgb]{0.73,0.40,0.53}{##1}}}
\expandafter\def\csname PY@tok@se\endcsname{\let\PY@bf=\textbf\def\PY@tc##1{\textcolor[rgb]{0.73,0.40,0.13}{##1}}}
\expandafter\def\csname PY@tok@sr\endcsname{\def\PY@tc##1{\textcolor[rgb]{0.73,0.40,0.53}{##1}}}
\expandafter\def\csname PY@tok@ss\endcsname{\def\PY@tc##1{\textcolor[rgb]{0.10,0.09,0.49}{##1}}}
\expandafter\def\csname PY@tok@sx\endcsname{\def\PY@tc##1{\textcolor[rgb]{0.00,0.50,0.00}{##1}}}
\expandafter\def\csname PY@tok@m\endcsname{\def\PY@tc##1{\textcolor[rgb]{0.40,0.40,0.40}{##1}}}
\expandafter\def\csname PY@tok@gh\endcsname{\let\PY@bf=\textbf\def\PY@tc##1{\textcolor[rgb]{0.00,0.00,0.50}{##1}}}
\expandafter\def\csname PY@tok@gu\endcsname{\let\PY@bf=\textbf\def\PY@tc##1{\textcolor[rgb]{0.50,0.00,0.50}{##1}}}
\expandafter\def\csname PY@tok@gd\endcsname{\def\PY@tc##1{\textcolor[rgb]{0.63,0.00,0.00}{##1}}}
\expandafter\def\csname PY@tok@gi\endcsname{\def\PY@tc##1{\textcolor[rgb]{0.00,0.63,0.00}{##1}}}
\expandafter\def\csname PY@tok@gr\endcsname{\def\PY@tc##1{\textcolor[rgb]{1.00,0.00,0.00}{##1}}}
\expandafter\def\csname PY@tok@ge\endcsname{\let\PY@it=\textit}
\expandafter\def\csname PY@tok@gs\endcsname{\let\PY@bf=\textbf}
\expandafter\def\csname PY@tok@gp\endcsname{\let\PY@bf=\textbf\def\PY@tc##1{\textcolor[rgb]{0.00,0.00,0.50}{##1}}}
\expandafter\def\csname PY@tok@go\endcsname{\def\PY@tc##1{\textcolor[rgb]{0.53,0.53,0.53}{##1}}}
\expandafter\def\csname PY@tok@gt\endcsname{\def\PY@tc##1{\textcolor[rgb]{0.00,0.27,0.87}{##1}}}
\expandafter\def\csname PY@tok@err\endcsname{\def\PY@bc##1{\setlength{\fboxsep}{0pt}\fcolorbox[rgb]{1.00,0.00,0.00}{1,1,1}{\strut ##1}}}
\expandafter\def\csname PY@tok@kc\endcsname{\let\PY@bf=\textbf\def\PY@tc##1{\textcolor[rgb]{0.00,0.50,0.00}{##1}}}
\expandafter\def\csname PY@tok@kd\endcsname{\let\PY@bf=\textbf\def\PY@tc##1{\textcolor[rgb]{0.00,0.50,0.00}{##1}}}
\expandafter\def\csname PY@tok@kn\endcsname{\let\PY@bf=\textbf\def\PY@tc##1{\textcolor[rgb]{0.00,0.50,0.00}{##1}}}
\expandafter\def\csname PY@tok@kr\endcsname{\let\PY@bf=\textbf\def\PY@tc##1{\textcolor[rgb]{0.00,0.50,0.00}{##1}}}
\expandafter\def\csname PY@tok@bp\endcsname{\def\PY@tc##1{\textcolor[rgb]{0.00,0.50,0.00}{##1}}}
\expandafter\def\csname PY@tok@fm\endcsname{\def\PY@tc##1{\textcolor[rgb]{0.00,0.00,1.00}{##1}}}
\expandafter\def\csname PY@tok@vc\endcsname{\def\PY@tc##1{\textcolor[rgb]{0.10,0.09,0.49}{##1}}}
\expandafter\def\csname PY@tok@vg\endcsname{\def\PY@tc##1{\textcolor[rgb]{0.10,0.09,0.49}{##1}}}
\expandafter\def\csname PY@tok@vi\endcsname{\def\PY@tc##1{\textcolor[rgb]{0.10,0.09,0.49}{##1}}}
\expandafter\def\csname PY@tok@vm\endcsname{\def\PY@tc##1{\textcolor[rgb]{0.10,0.09,0.49}{##1}}}
\expandafter\def\csname PY@tok@sa\endcsname{\def\PY@tc##1{\textcolor[rgb]{0.73,0.13,0.13}{##1}}}
\expandafter\def\csname PY@tok@sb\endcsname{\def\PY@tc##1{\textcolor[rgb]{0.73,0.13,0.13}{##1}}}
\expandafter\def\csname PY@tok@sc\endcsname{\def\PY@tc##1{\textcolor[rgb]{0.73,0.13,0.13}{##1}}}
\expandafter\def\csname PY@tok@dl\endcsname{\def\PY@tc##1{\textcolor[rgb]{0.73,0.13,0.13}{##1}}}
\expandafter\def\csname PY@tok@s2\endcsname{\def\PY@tc##1{\textcolor[rgb]{0.73,0.13,0.13}{##1}}}
\expandafter\def\csname PY@tok@sh\endcsname{\def\PY@tc##1{\textcolor[rgb]{0.73,0.13,0.13}{##1}}}
\expandafter\def\csname PY@tok@s1\endcsname{\def\PY@tc##1{\textcolor[rgb]{0.73,0.13,0.13}{##1}}}
\expandafter\def\csname PY@tok@mb\endcsname{\def\PY@tc##1{\textcolor[rgb]{0.40,0.40,0.40}{##1}}}
\expandafter\def\csname PY@tok@mf\endcsname{\def\PY@tc##1{\textcolor[rgb]{0.40,0.40,0.40}{##1}}}
\expandafter\def\csname PY@tok@mh\endcsname{\def\PY@tc##1{\textcolor[rgb]{0.40,0.40,0.40}{##1}}}
\expandafter\def\csname PY@tok@mi\endcsname{\def\PY@tc##1{\textcolor[rgb]{0.40,0.40,0.40}{##1}}}
\expandafter\def\csname PY@tok@il\endcsname{\def\PY@tc##1{\textcolor[rgb]{0.40,0.40,0.40}{##1}}}
\expandafter\def\csname PY@tok@mo\endcsname{\def\PY@tc##1{\textcolor[rgb]{0.40,0.40,0.40}{##1}}}
\expandafter\def\csname PY@tok@ch\endcsname{\let\PY@it=\textit\def\PY@tc##1{\textcolor[rgb]{0.25,0.50,0.50}{##1}}}
\expandafter\def\csname PY@tok@cm\endcsname{\let\PY@it=\textit\def\PY@tc##1{\textcolor[rgb]{0.25,0.50,0.50}{##1}}}
\expandafter\def\csname PY@tok@cpf\endcsname{\let\PY@it=\textit\def\PY@tc##1{\textcolor[rgb]{0.25,0.50,0.50}{##1}}}
\expandafter\def\csname PY@tok@c1\endcsname{\let\PY@it=\textit\def\PY@tc##1{\textcolor[rgb]{0.25,0.50,0.50}{##1}}}
\expandafter\def\csname PY@tok@cs\endcsname{\let\PY@it=\textit\def\PY@tc##1{\textcolor[rgb]{0.25,0.50,0.50}{##1}}}

\def\PYZbs{\char`\\}
\def\PYZus{\char`\_}
\def\PYZob{\char`\{}
\def\PYZcb{\char`\}}
\def\PYZca{\char`\^}
\def\PYZam{\char`\&}
\def\PYZlt{\char`\<}
\def\PYZgt{\char`\>}
\def\PYZsh{\char`\#}
\def\PYZpc{\char`\%}
\def\PYZdl{\char`\$}
\def\PYZhy{\char`\-}
\def\PYZsq{\char`\'}
\def\PYZdq{\char`\"}
\def\PYZti{\char`\~}
% for compatibility with earlier versions
\def\PYZat{@}
\def\PYZlb{[}
\def\PYZrb{]}
\makeatother


    % Exact colors from NB
    \definecolor{incolor}{rgb}{0.0, 0.0, 0.5}
    \definecolor{outcolor}{rgb}{0.545, 0.0, 0.0}



    
    % Prevent overflowing lines due to hard-to-break entities
    \sloppy 
    % Setup hyperref package
    \hypersetup{
      breaklinks=true,  % so long urls are correctly broken across lines
      colorlinks=true,
      urlcolor=urlcolor,
      linkcolor=linkcolor,
      citecolor=citecolor,
      }
    % Slightly bigger margins than the latex defaults
    
    \geometry{verbose,tmargin=1in,bmargin=1in,lmargin=1in,rmargin=1in}
    
    

    \begin{document}
    
    
    \maketitle
    
    

    
    \hypertarget{exercuxedcios-4.2-4.3-e-4.4-do-livro-texto}{%
\section{Exercícios 4.2, 4.3 e 4.4 do livro
texto}\label{exercuxedcios-4.2-4.3-e-4.4-do-livro-texto}}

    2018/3 Aluno: Pedro Bandeira de Mello Martins Disciplina: CPE773 -
Otimização Convexa Professor: Wallace A. Martins PEE/COPPE - UFRJ

    Minimizar \(f(x)\) no intervalo dado com incerteza menor que \(10^{-5}\)
com os métodos:

\begin{enumerate}
\def\labelenumi{\arabic{enumi}.}
\tightlist
\item
  Fibonacci Search
\item
  Golden-Section Search
\item
  Quadratic Interpolaton Method
\item
  Cubic Interpolation Method
\item
  Davies, Swann and Campey algorithm
\item
  Backtracking Line Search
\item
  Brute Force (implementação do scipy)
\end{enumerate}

O algoritmo de força bruta foi utilizado para compararmos a quantidade
necessária de avaliações para se chegar ao mesmo resultado.

    Os pacotes utilizados nesses exercícios são:

    \begin{Verbatim}[commandchars=\\\{\}]
{\color{incolor}In [{\color{incolor}1}]:} \PY{k+kn}{import} \PY{n+nn}{sys}
        \PY{k}{if} \PY{l+s+s1}{\PYZsq{}}\PY{l+s+s1}{..}\PY{l+s+s1}{\PYZsq{}} \PY{o+ow}{not} \PY{o+ow}{in} \PY{n}{sys}\PY{o}{.}\PY{n}{path}\PY{p}{:}
            \PY{n}{sys}\PY{o}{.}\PY{n}{path}\PY{o}{.}\PY{n}{append}\PY{p}{(}\PY{l+s+s1}{\PYZsq{}}\PY{l+s+s1}{..}\PY{l+s+s1}{\PYZsq{}}\PY{p}{)}
        \PY{k+kn}{import} \PY{n+nn}{matplotlib}\PY{n+nn}{.}\PY{n+nn}{pyplot} \PY{k}{as} \PY{n+nn}{plt}
        \PY{k+kn}{from} \PY{n+nn}{matplotlib} \PY{k}{import} \PY{n}{cm}
        \PY{k+kn}{from} \PY{n+nn}{mpl\PYZus{}toolkits}\PY{n+nn}{.}\PY{n+nn}{mplot3d} \PY{k}{import} \PY{n}{Axes3D}
        \PY{k+kn}{import} \PY{n+nn}{numpy} \PY{k}{as} \PY{n+nn}{np}
        \PY{k+kn}{import} \PY{n+nn}{pandas} \PY{k}{as} \PY{n+nn}{pd}
        \PY{k+kn}{import} \PY{n+nn}{time}
        \PY{k+kn}{from} \PY{n+nn}{copy} \PY{k}{import} \PY{n}{copy}
        
        \PY{k+kn}{from} \PY{n+nn}{scipy}\PY{n+nn}{.}\PY{n+nn}{optimize} \PY{k}{import} \PY{n}{brute}
        \PY{k+kn}{from} \PY{n+nn}{functions} \PY{k}{import} \PY{n}{order5\PYZus{}polynomial}\PY{p}{,} \PY{n}{logarithmic}\PY{p}{,} \PY{n}{sinoid}\PY{p}{,} \PY{n}{order4\PYZus{}polynomial}
        \PY{k+kn}{from} \PY{n+nn}{functions} \PY{k}{import} \PY{n}{functionObj}
        \PY{k+kn}{from} \PY{n+nn}{models}\PY{n+nn}{.}\PY{n+nn}{optimizers} \PY{k}{import} \PY{n}{DichotomousSearch}\PY{p}{,} \PYZbs{}
                                        \PY{n}{FibonacciSearch}\PY{p}{,} \PY{n}{GoldenSectionSearch}\PY{p}{,} \PYZbs{}
                                        \PY{n}{QuadraticInterpolationSearch}\PY{p}{,} \PYZbs{}
                                        \PY{n}{CubicInterpolation}\PY{p}{,} \PY{n}{DaviesSwannCampey}\PY{p}{,} \PYZbs{}
                                        \PY{n}{BacktrackingLineSearch}\PY{p}{,} \PY{n}{InexactLineSearch}
\end{Verbatim}


    Para os exercícios 4.2, 4.3 e 4.4, a função do arquivo
\url{run_exercises.py} é rodada, onde se entrega a função a ser
minimizada pelos métodos acima e ela retorna um \emph{dataframe} com
todas as informações obtidas durante as minimizações.

    \begin{Verbatim}[commandchars=\\\{\}]
{\color{incolor}In [{\color{incolor}2}]:} \PY{k+kn}{from} \PY{n+nn}{run\PYZus{}exercises} \PY{k}{import} \PY{n}{run\PYZus{}exercise}
\end{Verbatim}


    Os gráficos são gerados pela função:

    \begin{Verbatim}[commandchars=\\\{\}]
{\color{incolor}In [{\color{incolor}3}]:} \PY{k}{def} \PY{n+nf}{show\PYZus{}chart}\PY{p}{(}\PY{n}{df}\PY{p}{)}\PY{p}{:}
            \PY{n}{fig}\PY{p}{,} \PY{n}{axes} \PY{o}{=} \PY{n}{plt}\PY{o}{.}\PY{n}{subplots}\PY{p}{(}\PY{l+m+mi}{4}\PY{p}{,} \PY{l+m+mi}{2}\PY{p}{,} \PY{n}{figsize} \PY{o}{=} \PY{p}{(}\PY{l+m+mi}{13}\PY{p}{,} \PY{l+m+mi}{10}\PY{p}{)}\PY{p}{)}
            \PY{k}{for} \PY{n}{algorithm}\PY{p}{,} \PY{n}{ax} \PY{o+ow}{in} \PY{n+nb}{zip}\PY{p}{(}\PY{n}{df}\PY{o}{.}\PY{n}{index}\PY{p}{,} \PY{n}{axes}\PY{o}{.}\PY{n}{flatten}\PY{p}{(}\PY{p}{)}\PY{p}{)}\PY{p}{:}
                \PY{n}{ax}\PY{o}{.}\PY{n}{plot}\PY{p}{(}\PY{n+nb}{range}\PY{p}{(}\PY{l+m+mi}{1}\PY{p}{,} \PY{n}{df}\PY{p}{[}\PY{l+s+s1}{\PYZsq{}}\PY{l+s+s1}{fevals}\PY{l+s+s1}{\PYZsq{}}\PY{p}{]}\PY{p}{[}\PY{n}{algorithm}\PY{p}{]} \PY{o}{+} \PY{l+m+mi}{1}\PY{p}{)}\PY{p}{,} \PY{n}{df}\PY{p}{[}\PY{l+s+s1}{\PYZsq{}}\PY{l+s+s1}{all\PYZus{}evals}\PY{l+s+s1}{\PYZsq{}}\PY{p}{]}\PY{p}{[}\PY{n}{algorithm}\PY{p}{]}\PY{p}{)}
                \PY{n}{ax}\PY{o}{.}\PY{n}{set\PYZus{}title}\PY{p}{(}\PY{n}{algorithm}\PY{p}{)}
                \PY{n}{ax}\PY{o}{.}\PY{n}{ticklabel\PYZus{}format}\PY{p}{(}\PY{n}{axis} \PY{o}{=} \PY{l+s+s1}{\PYZsq{}}\PY{l+s+s1}{y}\PY{l+s+s1}{\PYZsq{}}\PY{p}{,} \PY{n}{style} \PY{o}{=} \PY{l+s+s1}{\PYZsq{}}\PY{l+s+s1}{plain}\PY{l+s+s1}{\PYZsq{}}\PY{p}{)}
                \PY{n}{ax}\PY{o}{.}\PY{n}{set\PYZus{}xlabel}\PY{p}{(}\PY{l+s+s1}{\PYZsq{}}\PY{l+s+s1}{Function evaluations}\PY{l+s+s1}{\PYZsq{}}\PY{p}{)}
                \PY{n}{ax}\PY{o}{.}\PY{n}{set\PYZus{}ylabel}\PY{p}{(}\PY{l+s+s1}{\PYZsq{}}\PY{l+s+s1}{\PYZdl{}f(x)\PYZdl{}}\PY{l+s+s1}{\PYZsq{}}\PY{p}{)}
            \PY{n}{plt}\PY{o}{.}\PY{n}{tight\PYZus{}layout}\PY{p}{(}\PY{p}{)}
            \PY{n}{plt}\PY{o}{.}\PY{n}{show}\PY{p}{(}\PY{p}{)}
\end{Verbatim}


    \hypertarget{exercuxedcio-4.2}{%
\subsection{Exercício 4.2}\label{exercuxedcio-4.2}}

\(f(x) = -5x^5 + 4x^4 - 12x^3 + 11x^2 - 2x + 1\)

    \begin{Verbatim}[commandchars=\\\{\}]
{\color{incolor}In [{\color{incolor}4}]:} \PY{n}{results\PYZus{}42} \PY{o}{=} \PY{n}{run\PYZus{}exercise}\PY{p}{(}\PY{n}{order5\PYZus{}polynomial}\PY{p}{,} \PY{n}{f\PYZus{}string} \PY{o}{=} \PY{l+s+s1}{\PYZsq{}}\PY{l+s+s1}{\PYZdl{}f(x) = \PYZhy{}5x\PYZca{}5+4x\PYZca{}4\PYZhy{}12x\PYZca{}3+11x\PYZca{}2\PYZhy{}2x+1\PYZdl{}}\PY{l+s+s1}{\PYZsq{}}\PY{p}{,} \PY{n}{interval} \PY{o}{=} \PY{p}{[}\PY{o}{\PYZhy{}}\PY{l+m+mf}{0.5}\PY{p}{,} \PY{l+m+mf}{0.5}\PY{p}{]}\PY{p}{)}
\end{Verbatim}


    \begin{center}
    \adjustimage{max size={0.9\linewidth}{0.9\paperheight}}{output_10_0.png}
    \end{center}
    { \hspace*{\fill} \\}
    
    \hypertarget{resultados}{%
\subsubsection{Resultados}\label{resultados}}

    \begin{Verbatim}[commandchars=\\\{\}]
{\color{incolor}In [{\color{incolor}5}]:} \PY{n}{results\PYZus{}42}\PY{p}{[}\PY{p}{[}\PY{l+s+s1}{\PYZsq{}}\PY{l+s+s1}{best\PYZus{}f}\PY{l+s+s1}{\PYZsq{}}\PY{p}{,} \PY{l+s+s1}{\PYZsq{}}\PY{l+s+s1}{best\PYZus{}x}\PY{l+s+s1}{\PYZsq{}}\PY{p}{,} \PY{l+s+s1}{\PYZsq{}}\PY{l+s+s1}{fevals}\PY{l+s+s1}{\PYZsq{}}\PY{p}{,} \PY{l+s+s1}{\PYZsq{}}\PY{l+s+s1}{run\PYZus{}time (s)}\PY{l+s+s1}{\PYZsq{}}\PY{p}{]}\PY{p}{]}
\end{Verbatim}


\begin{Verbatim}[commandchars=\\\{\}]
{\color{outcolor}Out[{\color{outcolor}5}]:}                                       best\_f    best\_x fevals  run\_time (s)
        Brute Force                         0.897633  0.109838     43      0.013442
        Dichotomous Search                  0.897633  0.109862     34      0.001680
        Fibonacci Search                    0.897633  0.109860      3      0.000207
        Golden-Section Search               0.897633  0.109860      3      0.000160
        Quadratic Interpolation Method      0.897633  0.109860      5      0.000275
        Cubic interpolation Method          0.897633  0.109860      5      0.001358
        Davies, Swann and Campey Algorithm  0.897633  0.109861      9      0.001077
        Backtracking Line Search            0.897633  0.109860      6      0.002515
\end{Verbatim}
            
    \hypertarget{eficiuxeancia-computacional-em-avaliauxe7uxf5es-de-funuxe7uxf5es-e-tempo-de-execuuxe7uxe3o.}{%
\subsubsection{Eficiência computacional em avaliações de funções e tempo
de
execução.}\label{eficiuxeancia-computacional-em-avaliauxe7uxf5es-de-funuxe7uxf5es-e-tempo-de-execuuxe7uxe3o.}}

    \begin{Verbatim}[commandchars=\\\{\}]
{\color{incolor}In [{\color{incolor}6}]:} \PY{n}{fig}\PY{p}{,} \PY{n}{axes} \PY{o}{=} \PY{n}{plt}\PY{o}{.}\PY{n}{subplots}\PY{p}{(}\PY{l+m+mi}{1}\PY{p}{,}\PY{l+m+mi}{2}\PY{p}{,} \PY{n}{figsize}\PY{o}{=}\PY{p}{(}\PY{l+m+mi}{13}\PY{p}{,}\PY{l+m+mi}{5}\PY{p}{)}\PY{p}{)}
        \PY{n}{fig}\PY{o}{.}\PY{n}{suptitle}\PY{p}{(}\PY{l+s+s1}{\PYZsq{}}\PY{l+s+s1}{\PYZdl{}f(x) = \PYZhy{}5x\PYZca{}5+4x\PYZca{}4\PYZhy{}12x\PYZca{}3+11x\PYZca{}2\PYZhy{}2x+1\PYZdl{}}\PY{l+s+s1}{\PYZsq{}}\PY{p}{)}
        \PY{n}{results\PYZus{}42}\PY{p}{[}\PY{l+s+s1}{\PYZsq{}}\PY{l+s+s1}{fevals}\PY{l+s+s1}{\PYZsq{}}\PY{p}{]}\PY{o}{.}\PY{n}{plot}\PY{o}{.}\PY{n}{bar}\PY{p}{(}\PY{n}{title} \PY{o}{=} \PY{l+s+s1}{\PYZsq{}}\PY{l+s+s1}{Function evals}\PY{l+s+s1}{\PYZsq{}}\PY{p}{,} \PY{n}{ax}\PY{o}{=} \PY{n}{axes}\PY{p}{[}\PY{l+m+mi}{0}\PY{p}{]}\PY{p}{)}
        \PY{p}{(}\PY{n}{results\PYZus{}42}\PY{p}{[}\PY{l+s+s1}{\PYZsq{}}\PY{l+s+s1}{run\PYZus{}time (s)}\PY{l+s+s1}{\PYZsq{}}\PY{p}{]}\PY{o}{*}\PY{l+m+mf}{1e3}\PY{p}{)}\PY{o}{.}\PY{n}{plot}\PY{o}{.}\PY{n}{bar}\PY{p}{(}\PY{n}{title} \PY{o}{=} \PY{l+s+s1}{\PYZsq{}}\PY{l+s+s1}{Run time (ms)}\PY{l+s+s1}{\PYZsq{}}\PY{p}{,} \PY{n}{ax}\PY{o}{=}\PY{n}{axes}\PY{p}{[}\PY{l+m+mi}{1}\PY{p}{]}\PY{p}{)}
        \PY{n}{plt}\PY{o}{.}\PY{n}{show}\PY{p}{(}\PY{p}{)}
\end{Verbatim}


    \begin{center}
    \adjustimage{max size={0.9\linewidth}{0.9\paperheight}}{output_14_0.png}
    \end{center}
    { \hspace*{\fill} \\}
    
    A tabela de resultados e os gráficos de eficiência computacional
demonstram que a força bruta é a minimização de maior custo
computacional tanto em número de avaliações de funções quanto em tempo
de execução.\\
Apesar da Dichotomous Search ser o segundo algoritmo a utilizar mais
avalições de funções, possui um tempo de execução menor do que a busca
por retrocesso (\emph{Backtracking Line Search}). Isso provavelmente
acontece porque o \emph{Backtracking Line Search} precisa estimar o
gradiente da função objetivo. O gradiente é calculado com a biblioteca
\href{https://github.com/HIPS/autograd}{autograd}.\\
Neste exercício a busca de seção dourada (\emph{Golden-Section Search})
foi o algoritmo de menor tempo de execução e avaliação de funções.

    \hypertarget{gruxe1ficos-de-fx-por-avaliauxe7uxf5es.}{%
\subsubsection{Gráficos de f(x) por
avaliações.}\label{gruxe1ficos-de-fx-por-avaliauxe7uxf5es.}}

    \begin{Verbatim}[commandchars=\\\{\}]
{\color{incolor}In [{\color{incolor}7}]:} \PY{n}{show\PYZus{}chart}\PY{p}{(}\PY{n}{results\PYZus{}42}\PY{p}{)}
\end{Verbatim}


    \begin{center}
    \adjustimage{max size={0.9\linewidth}{0.9\paperheight}}{output_17_0.png}
    \end{center}
    { \hspace*{\fill} \\}
    
    Pelos gráficos de \(f(x)\) por avaliações de funções, podemos ver que os
algoritmos \emph{Fibonacci Search}, \emph{Golden-Section Search},
\emph{Quadratic Interpolation Method}, \emph{Cubic Interpolation Method}
e \emph{Backtracking Line Search} encontraram o resultado logo na
primeira avaliação de função, porém continuaram a rodar até entregarem o
resultado. Isso acontece visto que a condição de parada deve ser
encontrada, o que não necessariamente é na primeira interação que
encontramos o mínimo.

    \hypertarget{exercuxedcio-4.3}{%
\subsection{Exercício 4.3}\label{exercuxedcio-4.3}}

\(f(x) = \ln ^2 (x-2) + \ln ^2(10-x) - x^{0.2}\)

    \begin{Verbatim}[commandchars=\\\{\}]
{\color{incolor}In [{\color{incolor}8}]:} \PY{n}{results\PYZus{}43} \PY{o}{=} \PY{n}{run\PYZus{}exercise}\PY{p}{(}\PY{n}{logarithmic}\PY{p}{,} \PY{n}{f\PYZus{}string} \PY{o}{=} \PY{l+s+s1}{\PYZsq{}}\PY{l+s+s1}{\PYZdl{}f(x) = }\PY{l+s+s1}{\PYZbs{}}\PY{l+s+s1}{ln \PYZca{}2 (x\PYZhy{}2) + }\PY{l+s+s1}{\PYZbs{}}\PY{l+s+s1}{ln \PYZca{}2(10\PYZhy{}x) \PYZhy{} x\PYZca{}}\PY{l+s+si}{\PYZob{}0.2\PYZcb{}}\PY{l+s+s1}{\PYZdl{}}\PY{l+s+s1}{\PYZsq{}}\PY{p}{,} 
                                  \PY{n}{seed} \PY{o}{=}  \PY{l+m+mi}{9}\PY{p}{,} 
                                  \PY{n}{textpos} \PY{o}{=} \PY{p}{(}\PY{l+m+mi}{12}\PY{p}{,}\PY{l+m+mi}{40}\PY{p}{)}\PY{p}{,}
                                  \PY{n}{interval} \PY{o}{=} \PY{p}{[}\PY{l+m+mi}{6}\PY{p}{,} \PY{l+m+mf}{9.9}\PY{p}{]}\PY{p}{)}
\end{Verbatim}


    \begin{center}
    \adjustimage{max size={0.9\linewidth}{0.9\paperheight}}{output_20_0.png}
    \end{center}
    { \hspace*{\fill} \\}
    
    \hypertarget{resultados}{%
\subsubsection{Resultados}\label{resultados}}

    \begin{Verbatim}[commandchars=\\\{\}]
{\color{incolor}In [{\color{incolor}9}]:} \PY{n}{results\PYZus{}43}\PY{p}{[}\PY{p}{[}\PY{l+s+s1}{\PYZsq{}}\PY{l+s+s1}{best\PYZus{}f}\PY{l+s+s1}{\PYZsq{}}\PY{p}{,} \PY{l+s+s1}{\PYZsq{}}\PY{l+s+s1}{best\PYZus{}x}\PY{l+s+s1}{\PYZsq{}}\PY{p}{,} \PY{l+s+s1}{\PYZsq{}}\PY{l+s+s1}{fevals}\PY{l+s+s1}{\PYZsq{}}\PY{p}{,} \PY{l+s+s1}{\PYZsq{}}\PY{l+s+s1}{run\PYZus{}time (s)}\PY{l+s+s1}{\PYZsq{}}\PY{p}{]}\PY{p}{]}
\end{Verbatim}


\begin{Verbatim}[commandchars=\\\{\}]
{\color{outcolor}Out[{\color{outcolor}9}]:}                                       best\_f    best\_x fevals  run\_time (s)
        Brute Force                         2.133838  8.501589     49      0.007866
        Dichotomous Search                  2.133838  8.501585     38      0.002073
        Fibonacci Search                    2.133838  8.501586      3      0.000308
        Golden-Section Search               2.133838  8.501586      3      0.000176
        Quadratic Interpolation Method      2.133838  8.501587      5      0.000296
        Cubic interpolation Method          2.133838  8.501587     10      0.003075
        Davies, Swann and Campey Algorithm  2.133838  8.501585     15      0.001615
        Backtracking Line Search            2.133838  8.501587      2      0.001401
\end{Verbatim}
            
    \hypertarget{eficiuxeancia-computacional-em-avaliauxe7uxf5es-de-funuxe7uxf5es-e-tempo-de-execuuxe7uxe3o.}{%
\subsubsection{Eficiência computacional em avaliações de funções e tempo
de
execução.}\label{eficiuxeancia-computacional-em-avaliauxe7uxf5es-de-funuxe7uxf5es-e-tempo-de-execuuxe7uxe3o.}}

    \begin{Verbatim}[commandchars=\\\{\}]
{\color{incolor}In [{\color{incolor}10}]:} \PY{n}{fig}\PY{p}{,} \PY{n}{axes} \PY{o}{=} \PY{n}{plt}\PY{o}{.}\PY{n}{subplots}\PY{p}{(}\PY{l+m+mi}{1}\PY{p}{,}\PY{l+m+mi}{2}\PY{p}{,} \PY{n}{figsize}\PY{o}{=}\PY{p}{(}\PY{l+m+mi}{13}\PY{p}{,}\PY{l+m+mi}{5}\PY{p}{)}\PY{p}{)}
         \PY{n}{fig}\PY{o}{.}\PY{n}{suptitle}\PY{p}{(}\PY{l+s+s1}{\PYZsq{}}\PY{l+s+s1}{\PYZdl{}f(x) = }\PY{l+s+s1}{\PYZbs{}}\PY{l+s+s1}{ln \PYZca{}2 (x\PYZhy{}2) + }\PY{l+s+s1}{\PYZbs{}}\PY{l+s+s1}{ln \PYZca{}2(10\PYZhy{}x) \PYZhy{} x\PYZca{}}\PY{l+s+si}{\PYZob{}0.2\PYZcb{}}\PY{l+s+s1}{\PYZdl{}}\PY{l+s+s1}{\PYZsq{}}\PY{p}{)}
         \PY{n}{results\PYZus{}43}\PY{p}{[}\PY{l+s+s1}{\PYZsq{}}\PY{l+s+s1}{fevals}\PY{l+s+s1}{\PYZsq{}}\PY{p}{]}\PY{o}{.}\PY{n}{plot}\PY{o}{.}\PY{n}{bar}\PY{p}{(}\PY{n}{title} \PY{o}{=} \PY{l+s+s1}{\PYZsq{}}\PY{l+s+s1}{Function evals}\PY{l+s+s1}{\PYZsq{}}\PY{p}{,} \PY{n}{ax}\PY{o}{=} \PY{n}{axes}\PY{p}{[}\PY{l+m+mi}{0}\PY{p}{]}\PY{p}{)}
         \PY{p}{(}\PY{n}{results\PYZus{}43}\PY{p}{[}\PY{l+s+s1}{\PYZsq{}}\PY{l+s+s1}{run\PYZus{}time (s)}\PY{l+s+s1}{\PYZsq{}}\PY{p}{]}\PY{o}{*}\PY{l+m+mf}{1e3}\PY{p}{)}\PY{o}{.}\PY{n}{plot}\PY{o}{.}\PY{n}{bar}\PY{p}{(}\PY{n}{title} \PY{o}{=} \PY{l+s+s1}{\PYZsq{}}\PY{l+s+s1}{Run time (ms)}\PY{l+s+s1}{\PYZsq{}}\PY{p}{,} \PY{n}{ax}\PY{o}{=}\PY{n}{axes}\PY{p}{[}\PY{l+m+mi}{1}\PY{p}{]}\PY{p}{)}
         \PY{n}{plt}\PY{o}{.}\PY{n}{show}\PY{p}{(}\PY{p}{)}
\end{Verbatim}


    \begin{center}
    \adjustimage{max size={0.9\linewidth}{0.9\paperheight}}{output_24_0.png}
    \end{center}
    { \hspace*{\fill} \\}
    
    No exercício 4.3 observa-se que a força bruta continua sendo o algoritmo
de menor eficiência computacional. Os algoritmos que utilizam gradiente
da função também se mostraram menos eficientes computacionalmente em
tempo de execução, apesar de terem efetuado poucas avaliações de
funções.\\
Como no exercício 4.2, o Dichotomous Search permaneceu em segunda pior
avaliação em \emph{Function Evals} e \emph{Run time (ms)}.

    \hypertarget{gruxe1ficos-de-fx-por-avaliauxe7uxf5es.}{%
\subsubsection{Gráficos de f(x) por
avaliações.}\label{gruxe1ficos-de-fx-por-avaliauxe7uxf5es.}}

    \begin{Verbatim}[commandchars=\\\{\}]
{\color{incolor}In [{\color{incolor}11}]:} \PY{n}{show\PYZus{}chart}\PY{p}{(}\PY{n}{results\PYZus{}43}\PY{p}{)}
\end{Verbatim}


    \begin{center}
    \adjustimage{max size={0.9\linewidth}{0.9\paperheight}}{output_27_0.png}
    \end{center}
    { \hspace*{\fill} \\}
    
    Nos gráficos de \(f(x)\) por avaliações do exercício 4.3, é interessante
notar que o \emph{Dichotomous Search} obteve o mesmo comportamento do
exercício 4.2, apesar de atingir valores diferentes de \(f(x)\). Os
algoritmos \emph{Fibonacci Search}, \emph{Golden-Section Search},
\emph{Cubic Interpolation Method} e \emph{Backtracking Line Search}
atingiram o mínimo global na primeira iteração. Apesar do algoritmo
\emph{Quadratic Interpolation Method} ter variado mais do que os já
citados, ele permaneceu dentrou da tolerância de erro durante as 5
iterações.

    \hypertarget{exercuxedcio-4.4}{%
\subsection{Exercício 4.4}\label{exercuxedcio-4.4}}

\(f(x) = -3x\sin 0.75 x + e ^{-2x}\)

    \begin{Verbatim}[commandchars=\\\{\}]
{\color{incolor}In [{\color{incolor}12}]:} \PY{n}{results\PYZus{}44} \PY{o}{=} \PY{n}{run\PYZus{}exercise}\PY{p}{(}\PY{n}{sinoid}\PY{p}{,} \PY{n}{f\PYZus{}string} \PY{o}{=} \PY{l+s+s1}{\PYZsq{}}\PY{l+s+s1}{\PYZdl{}f(x) = \PYZhy{}3x}\PY{l+s+s1}{\PYZbs{}}\PY{l+s+s1}{sin 0.75 x + e \PYZca{}}\PY{l+s+s1}{\PYZob{}}\PY{l+s+s1}{\PYZhy{}2x\PYZcb{}\PYZdl{}}\PY{l+s+s1}{\PYZsq{}}\PY{p}{,} 
                                   \PY{n}{seed} \PY{o}{=}  \PY{l+m+mi}{9}\PY{p}{,} 
                                   \PY{n}{interval} \PY{o}{=} \PY{p}{[}\PY{l+m+mi}{0}\PY{p}{,} \PY{l+m+mi}{2}\PY{o}{*}\PY{n}{np}\PY{o}{.}\PY{n}{pi}\PY{p}{]}\PY{p}{)}
\end{Verbatim}


    \begin{center}
    \adjustimage{max size={0.9\linewidth}{0.9\paperheight}}{output_30_0.png}
    \end{center}
    { \hspace*{\fill} \\}
    
    \hypertarget{resultados}{%
\subsubsection{Resultados}\label{resultados}}

    \begin{Verbatim}[commandchars=\\\{\}]
{\color{incolor}In [{\color{incolor}13}]:} \PY{n}{results\PYZus{}44}\PY{p}{[}\PY{p}{[}\PY{l+s+s1}{\PYZsq{}}\PY{l+s+s1}{best\PYZus{}f}\PY{l+s+s1}{\PYZsq{}}\PY{p}{,} \PY{l+s+s1}{\PYZsq{}}\PY{l+s+s1}{best\PYZus{}x}\PY{l+s+s1}{\PYZsq{}}\PY{p}{,} \PY{l+s+s1}{\PYZsq{}}\PY{l+s+s1}{fevals}\PY{l+s+s1}{\PYZsq{}}\PY{p}{,} \PY{l+s+s1}{\PYZsq{}}\PY{l+s+s1}{run\PYZus{}time (s)}\PY{l+s+s1}{\PYZsq{}}\PY{p}{]}\PY{p}{]}
\end{Verbatim}


\begin{Verbatim}[commandchars=\\\{\}]
{\color{outcolor}Out[{\color{outcolor}13}]:}                                       best\_f    best\_x fevals  run\_time (s)
         Brute Force                        -7.274358  2.706459     45      0.005145
         Dichotomous Search                 -7.274358  2.706477     40      0.002044
         Fibonacci Search                   -7.274358  2.706476      3      0.000214
         Golden-Section Search              -7.274358  2.706476      3      0.000165
         Quadratic Interpolation Method     -7.274358  2.706476      5      0.000280
         Cubic interpolation Method         -7.274358  2.706476      5      0.000867
         Davies, Swann and Campey Algorithm -7.274358  2.706475     12      0.001239
         Backtracking Line Search           -7.274358  2.706476      5      0.001361
\end{Verbatim}
            
    \hypertarget{eficiuxeancia-computacional-em-avaliauxe7uxf5es-de-funuxe7uxf5es-e-tempo-de-execuuxe7uxe3o.}{%
\subsubsection{Eficiência computacional em avaliações de funções e tempo
de
execução.}\label{eficiuxeancia-computacional-em-avaliauxe7uxf5es-de-funuxe7uxf5es-e-tempo-de-execuuxe7uxe3o.}}

    \begin{Verbatim}[commandchars=\\\{\}]
{\color{incolor}In [{\color{incolor}14}]:} \PY{n}{fig}\PY{p}{,} \PY{n}{axes} \PY{o}{=} \PY{n}{plt}\PY{o}{.}\PY{n}{subplots}\PY{p}{(}\PY{l+m+mi}{1}\PY{p}{,}\PY{l+m+mi}{2}\PY{p}{,} \PY{n}{figsize}\PY{o}{=}\PY{p}{(}\PY{l+m+mi}{13}\PY{p}{,}\PY{l+m+mi}{5}\PY{p}{)}\PY{p}{)}
         \PY{n}{fig}\PY{o}{.}\PY{n}{suptitle}\PY{p}{(}\PY{l+s+s1}{\PYZsq{}}\PY{l+s+s1}{\PYZdl{}f(x) = \PYZhy{}3x}\PY{l+s+s1}{\PYZbs{}}\PY{l+s+s1}{sin 0.75 x + e \PYZca{}}\PY{l+s+s1}{\PYZob{}}\PY{l+s+s1}{\PYZhy{}2x\PYZcb{}\PYZdl{}}\PY{l+s+s1}{\PYZsq{}}\PY{p}{)}
         \PY{n}{results\PYZus{}44}\PY{p}{[}\PY{l+s+s1}{\PYZsq{}}\PY{l+s+s1}{fevals}\PY{l+s+s1}{\PYZsq{}}\PY{p}{]}\PY{o}{.}\PY{n}{plot}\PY{o}{.}\PY{n}{bar}\PY{p}{(}\PY{n}{title} \PY{o}{=} \PY{l+s+s1}{\PYZsq{}}\PY{l+s+s1}{Function evals}\PY{l+s+s1}{\PYZsq{}}\PY{p}{,} \PY{n}{ax}\PY{o}{=} \PY{n}{axes}\PY{p}{[}\PY{l+m+mi}{0}\PY{p}{]}\PY{p}{)}
         \PY{p}{(}\PY{n}{results\PYZus{}44}\PY{p}{[}\PY{l+s+s1}{\PYZsq{}}\PY{l+s+s1}{run\PYZus{}time (s)}\PY{l+s+s1}{\PYZsq{}}\PY{p}{]}\PY{o}{*}\PY{l+m+mf}{1e3}\PY{p}{)}\PY{o}{.}\PY{n}{plot}\PY{o}{.}\PY{n}{bar}\PY{p}{(}\PY{n}{title} \PY{o}{=} \PY{l+s+s1}{\PYZsq{}}\PY{l+s+s1}{Run time (ms)}\PY{l+s+s1}{\PYZsq{}}\PY{p}{,} \PY{n}{ax}\PY{o}{=}\PY{n}{axes}\PY{p}{[}\PY{l+m+mi}{1}\PY{p}{]}\PY{p}{)}
         \PY{n}{plt}\PY{o}{.}\PY{n}{show}\PY{p}{(}\PY{p}{)}
\end{Verbatim}


    \begin{center}
    \adjustimage{max size={0.9\linewidth}{0.9\paperheight}}{output_34_0.png}
    \end{center}
    { \hspace*{\fill} \\}
    
    Os resultados obtidos no exercício 4.4 foram próximos dos resultados do
exercício 4.3. Após avaliação dos três primeiros exercícios, a
\emph{Golden-Section Search} se mostrou o melhor algoritmo de
minimização para os três problemas.

    \hypertarget{gruxe1ficos-de-fx-por-avaliauxe7uxf5es.}{%
\subsubsection{Gráficos de f(x) por
avaliações.}\label{gruxe1ficos-de-fx-por-avaliauxe7uxf5es.}}

    \begin{Verbatim}[commandchars=\\\{\}]
{\color{incolor}In [{\color{incolor}15}]:} \PY{n}{show\PYZus{}chart}\PY{p}{(}\PY{n}{results\PYZus{}44}\PY{p}{)}
\end{Verbatim}


    \begin{center}
    \adjustimage{max size={0.9\linewidth}{0.9\paperheight}}{output_37_0.png}
    \end{center}
    { \hspace*{\fill} \\}
    
    No exercício 4.4, o \emph{Dichotomous Search} apresentou um
comportamento diferente dos exercícios 4.2 e 4.3.

    \hypertarget{exercuxedcio-4.11}{%
\subsection{Exercício 4.11}\label{exercuxedcio-4.11}}

\(f(\mathbf{x}) = 0.7x^4_1 - 8x^2_1 + 6x^2_2 + \cos(x_1x_2)-8x_1\)

    \hypertarget{item-a}{%
\subsubsection{Item a)}\label{item-a}}

    \begin{Verbatim}[commandchars=\\\{\}]
{\color{incolor}In [{\color{incolor}16}]:} \PY{k+kn}{from} \PY{n+nn}{run\PYZus{}exercises} \PY{k}{import} \PY{n}{plot\PYZus{}surface}
         \PY{n}{plot\PYZus{}surface}\PY{p}{(}\PY{p}{)}
\end{Verbatim}


    \begin{center}
    \adjustimage{max size={0.9\linewidth}{0.9\paperheight}}{output_41_0.png}
    \end{center}
    { \hspace*{\fill} \\}
    
    \hypertarget{item-b}{%
\subsubsection{Item b)}\label{item-b}}

    \begin{Verbatim}[commandchars=\\\{\}]
{\color{incolor}In [{\color{incolor}17}]:} \PY{k+kn}{from} \PY{n+nn}{run\PYZus{}exercises} \PY{k}{import} \PY{n}{plot\PYZus{}contour}
         \PY{n}{plot\PYZus{}contour}\PY{p}{(}\PY{p}{)}
\end{Verbatim}


    \begin{center}
    \adjustimage{max size={0.9\linewidth}{0.9\paperheight}}{output_43_0.png}
    \end{center}
    { \hspace*{\fill} \\}
    
    \hypertarget{item-d}{%
\subsubsection{Item d)}\label{item-d}}

Para o item d, rodou-se os algoritmos de Backtracking e Fletcher's
Inexact Line Search. Os dois encontraram \(\alpha\)s a partir de
\(\mathbf{x}_0 = [-\pi, \pi]\) e \(\mathbf{d}_0 = [1.0, -1.3]\). A
solução de Fletcher encontrou um custo menor que a solução de
Backtracking, como visto na próxima célula.

    \begin{Verbatim}[commandchars=\\\{\}]
{\color{incolor}In [{\color{incolor}18}]:} \PY{k+kn}{from} \PY{n+nn}{run\PYZus{}exercises} \PY{k}{import} \PY{n}{plot\PYZus{}func\PYZus{}alpha}
         \PY{n}{x\PYZus{}0} \PY{o}{=} \PY{n}{np}\PY{o}{.}\PY{n}{array}\PY{p}{(}\PY{p}{[}\PY{o}{\PYZhy{}}\PY{n}{np}\PY{o}{.}\PY{n}{pi}\PY{p}{,} \PY{n}{np}\PY{o}{.}\PY{n}{pi}\PY{p}{]}\PY{p}{)}
         \PY{n}{d\PYZus{}0} \PY{o}{=} \PY{n}{np}\PY{o}{.}\PY{n}{array}\PY{p}{(}\PY{p}{[}\PY{l+m+mf}{1.0}\PY{p}{,} \PY{o}{\PYZhy{}}\PY{l+m+mf}{1.3}\PY{p}{]}\PY{p}{)}
         \PY{n}{func} \PY{o}{=} \PY{n}{functionObj}\PY{p}{(}\PY{n}{order4\PYZus{}polynomial}\PY{p}{)}
         
         \PY{n}{item\PYZus{}d\PYZus{}optimizer} \PY{o}{=} \PY{n}{InexactLineSearch}\PY{p}{(}\PY{n}{func}\PY{p}{,} \PY{n}{x\PYZus{}0}\PY{p}{,} \PY{n}{d\PYZus{}0}\PY{p}{)}
         \PY{n}{backtracking\PYZus{}opt} \PY{o}{=} \PY{n}{BacktrackingLineSearch}\PY{p}{(}\PY{n}{func}\PY{p}{,} \PY{n}{x\PYZus{}0}\PY{p}{,} \PY{n}{d\PYZus{}0}\PY{p}{)}
         \PY{n}{alpha\PYZus{}f}\PY{p}{,} \PY{n}{f0\PYZus{}f} \PY{o}{=} \PY{n}{item\PYZus{}d\PYZus{}optimizer}\PY{o}{.}\PY{n}{\PYZus{}line\PYZus{}search}\PY{p}{(}\PY{p}{)}
         \PY{n}{alpha\PYZus{}b}\PY{p}{,} \PY{n}{f0\PYZus{}b} \PY{o}{=} \PY{n}{backtracking\PYZus{}opt}\PY{o}{.}\PY{n}{\PYZus{}backtracking\PYZus{}line\PYZus{}search}\PY{p}{(}\PY{n}{func}\PY{o}{.}\PY{n}{grad}\PY{p}{(}\PY{n}{x\PYZus{}0}\PY{p}{)}\PY{p}{)}
         \PY{n+nb}{print}\PY{p}{(}\PY{l+s+s1}{\PYZsq{}}\PY{l+s+s1}{Inexact Line Search Methods:}\PY{l+s+s1}{\PYZsq{}}\PY{p}{)}
         \PY{n+nb}{print}\PY{p}{(}\PY{l+s+s1}{\PYZsq{}}\PY{l+s+s1}{ \PYZhy{} Fletcher solution}\PY{l+s+se}{\PYZbs{}n}\PY{l+s+s1}{   }\PY{l+s+s1}{\PYZsq{}}\PY{o}{+}\PY{l+s+sa}{u}\PY{l+s+s1}{\PYZsq{}}\PY{l+s+se}{\PYZbs{}u00B7}\PY{l+s+s1}{\PYZsq{}} \PY{o}{+} \PY{l+s+sa}{u}\PY{l+s+s1}{\PYZsq{}}\PY{l+s+se}{\PYZbs{}u03B1}\PY{l+s+s1}{\PYZsq{}}\PY{o}{+}\PY{l+s+s1}{\PYZsq{}}\PY{l+s+s1}{: }\PY{l+s+si}{\PYZpc{}.7f}\PY{l+s+se}{\PYZbs{}n}\PY{l+s+s1}{   }\PY{l+s+s1}{\PYZsq{}}\PY{o}{\PYZpc{}}\PY{k}{alpha\PYZus{}f} + u\PYZsq{}\PYZbs{}u00B7\PYZsq{} + \PYZsq{}f: \PYZpc{}.7f\PYZsq{}\PYZpc{}f0\PYZus{}f)
         \PY{n+nb}{print}\PY{p}{(}\PY{l+s+s1}{\PYZsq{}}\PY{l+s+s1}{ \PYZhy{} Backtracking solution}\PY{l+s+se}{\PYZbs{}n}\PY{l+s+s1}{   }\PY{l+s+s1}{\PYZsq{}}\PY{o}{+}\PY{l+s+sa}{u}\PY{l+s+s1}{\PYZsq{}}\PY{l+s+se}{\PYZbs{}u00B7}\PY{l+s+s1}{\PYZsq{}}\PY{o}{+} \PY{l+s+sa}{u}\PY{l+s+s1}{\PYZsq{}}\PY{l+s+se}{\PYZbs{}u03B1}\PY{l+s+s1}{\PYZsq{}}\PY{o}{+}\PY{l+s+s1}{\PYZsq{}}\PY{l+s+s1}{: }\PY{l+s+si}{\PYZpc{}.7f}\PY{l+s+se}{\PYZbs{}n}\PY{l+s+s1}{   }\PY{l+s+s1}{\PYZsq{}}\PY{o}{\PYZpc{}}\PY{k}{alpha\PYZus{}b} + u\PYZsq{}\PYZbs{}u00B7\PYZsq{} + \PYZsq{}f: \PYZpc{}.7f\PYZsq{}\PYZpc{}f0\PYZus{}b)
\end{Verbatim}


    \begin{Verbatim}[commandchars=\\\{\}]
Inexact Line Search Methods:
 - Fletcher solution
   ·α: 2.1191411
   ·f: 2.4014783
 - Backtracking solution
   ·α: 1.0000000
   ·f: 14.8198176

    \end{Verbatim}

    Observando o gráfico a seguir, pode-se ver que o \(\mathbf x\)
encontrado pela solução de Fletcher permaneceu na região
\(C = \{x | 0.0 \geq f(x) \leq 15.0\}\) mais próxima da região de mínimo
local \(C_{minlocal} = \{x | f(x) \leq 0.0 \}\). O Backtracking
encontrou uma solução na borda do nível
\(C_{15} = \{x | f(x) = 15.0\}\).

    \begin{Verbatim}[commandchars=\\\{\}]
{\color{incolor}In [{\color{incolor}19}]:} \PY{n}{plot\PYZus{}contour}\PY{p}{(}\PY{n}{x\PYZus{}0} \PY{o}{+} \PY{n}{alpha\PYZus{}f}\PY{o}{*}\PY{n}{d\PYZus{}0}\PY{p}{,} \PY{n}{x\PYZus{}0} \PY{o}{+} \PY{n}{alpha\PYZus{}b}\PY{o}{*}\PY{n}{d\PYZus{}0}\PY{p}{)}
\end{Verbatim}


    \begin{center}
    \adjustimage{max size={0.9\linewidth}{0.9\paperheight}}{output_47_0.png}
    \end{center}
    { \hspace*{\fill} \\}
    
    O gráfico a seguir demonstra que o Fletcher's Inexact line Search
encontrou um \(\alpha\) mínimo da função
\(f(\mathbf{x}_0 + \alpha \mathbf{d}_0)\) melhor do que o Backtracking
inexact line search.

    \begin{Verbatim}[commandchars=\\\{\}]
{\color{incolor}In [{\color{incolor}20}]:} \PY{n}{alphas} \PY{o}{=} \PY{n}{np}\PY{o}{.}\PY{n}{linspace}\PY{p}{(}\PY{l+m+mi}{0}\PY{p}{,} \PY{l+m+mf}{4.8332}\PY{p}{)}
         \PY{n}{plot\PYZus{}func\PYZus{}alpha}\PY{p}{(}\PY{n}{x\PYZus{}0}\PY{p}{,} \PY{n}{d\PYZus{}0}\PY{p}{,} \PY{n}{alphas}\PY{p}{,} \PY{n}{alpha\PYZus{}f}\PY{p}{,} \PY{n}{f0\PYZus{}f}\PY{p}{,} \PY{n}{alpha\PYZus{}b}\PY{p}{,} \PY{n}{f0\PYZus{}b}\PY{p}{)}
\end{Verbatim}


    \begin{center}
    \adjustimage{max size={0.9\linewidth}{0.9\paperheight}}{output_49_0.png}
    \end{center}
    { \hspace*{\fill} \\}
    
    \hypertarget{item-e}{%
\subsubsection{Item e)}\label{item-e}}

Para o item e), o Fletcher's Inexact Line Search encontrou uma solução
melhor do que no item d). Isso se deve pelo fato do vetor direção
\(\mathbf{d}_0\) apontar para o mínimo global da função custo. O
backtracking line search permaneceu encontrando o \(\alpha=1.0\), isso
se deve ao fato a busca inexata não ter entrado na condição do
\emph{while} na primeira iteração, permanecendo assim o \(\alpha\) como
fora inicializado no
\href{../models/optimizers/BacktrackingLineSearch.py}{código}.

    \begin{Verbatim}[commandchars=\\\{\}]
{\color{incolor}In [{\color{incolor}21}]:} \PY{n}{x\PYZus{}0} \PY{o}{=} \PY{n}{np}\PY{o}{.}\PY{n}{array}\PY{p}{(}\PY{p}{[}\PY{o}{\PYZhy{}}\PY{n}{np}\PY{o}{.}\PY{n}{pi}\PY{p}{,} \PY{n}{np}\PY{o}{.}\PY{n}{pi}\PY{p}{]}\PY{p}{)}
         \PY{n}{d\PYZus{}0} \PY{o}{=} \PY{n}{np}\PY{o}{.}\PY{n}{array}\PY{p}{(}\PY{p}{[}\PY{l+m+mf}{1.0}\PY{p}{,} \PY{o}{\PYZhy{}}\PY{l+m+mf}{1.1}\PY{p}{]}\PY{p}{)}
         \PY{n}{func} \PY{o}{=} \PY{n}{functionObj}\PY{p}{(}\PY{n}{order4\PYZus{}polynomial}\PY{p}{)}
         \PY{n}{item\PYZus{}d\PYZus{}optimizer} \PY{o}{=} \PY{n}{InexactLineSearch}\PY{p}{(}\PY{n}{func}\PY{p}{,} \PY{n}{x\PYZus{}0}\PY{p}{,} \PY{n}{d\PYZus{}0}\PY{p}{)}
         \PY{n}{backtracking\PYZus{}opt} \PY{o}{=} \PY{n}{BacktrackingLineSearch}\PY{p}{(}\PY{n}{func}\PY{p}{,} \PY{n}{x\PYZus{}0}\PY{p}{,} \PY{n}{d\PYZus{}0}\PY{p}{)}
         \PY{n}{alpha\PYZus{}f}\PY{p}{,} \PY{n}{f0\PYZus{}f} \PY{o}{=} \PY{n}{item\PYZus{}d\PYZus{}optimizer}\PY{o}{.}\PY{n}{\PYZus{}line\PYZus{}search}\PY{p}{(}\PY{p}{)}
         \PY{n}{alpha\PYZus{}b}\PY{p}{,} \PY{n}{f0\PYZus{}b} \PY{o}{=} \PY{n}{backtracking\PYZus{}opt}\PY{o}{.}\PY{n}{\PYZus{}backtracking\PYZus{}line\PYZus{}search}\PY{p}{(}\PY{n}{func}\PY{o}{.}\PY{n}{grad}\PY{p}{(}\PY{n}{x\PYZus{}0}\PY{p}{)}\PY{p}{)}
         \PY{n+nb}{print}\PY{p}{(}\PY{l+s+s1}{\PYZsq{}}\PY{l+s+s1}{Inexact Line Search Methods:}\PY{l+s+s1}{\PYZsq{}}\PY{p}{)}
         \PY{n+nb}{print}\PY{p}{(}\PY{l+s+s1}{\PYZsq{}}\PY{l+s+s1}{ \PYZhy{} Fletcher solution}\PY{l+s+se}{\PYZbs{}n}\PY{l+s+s1}{   }\PY{l+s+s1}{\PYZsq{}}\PY{o}{+}\PY{l+s+sa}{u}\PY{l+s+s1}{\PYZsq{}}\PY{l+s+se}{\PYZbs{}u00B7}\PY{l+s+s1}{\PYZsq{}} \PY{o}{+} \PY{l+s+sa}{u}\PY{l+s+s1}{\PYZsq{}}\PY{l+s+se}{\PYZbs{}u03B1}\PY{l+s+s1}{\PYZsq{}}\PY{o}{+}\PY{l+s+s1}{\PYZsq{}}\PY{l+s+s1}{: }\PY{l+s+si}{\PYZpc{}.7f}\PY{l+s+se}{\PYZbs{}n}\PY{l+s+s1}{   }\PY{l+s+s1}{\PYZsq{}}\PY{o}{\PYZpc{}}\PY{k}{alpha\PYZus{}f} + u\PYZsq{}\PYZbs{}u00B7\PYZsq{} + \PYZsq{}f: \PYZpc{}.7f\PYZsq{}\PYZpc{}f0\PYZus{}f)
         \PY{n+nb}{print}\PY{p}{(}\PY{l+s+s1}{\PYZsq{}}\PY{l+s+s1}{ \PYZhy{} Backtracking solution}\PY{l+s+se}{\PYZbs{}n}\PY{l+s+s1}{   }\PY{l+s+s1}{\PYZsq{}}\PY{o}{+}\PY{l+s+sa}{u}\PY{l+s+s1}{\PYZsq{}}\PY{l+s+se}{\PYZbs{}u00B7}\PY{l+s+s1}{\PYZsq{}}\PY{o}{+} \PY{l+s+sa}{u}\PY{l+s+s1}{\PYZsq{}}\PY{l+s+se}{\PYZbs{}u03B1}\PY{l+s+s1}{\PYZsq{}}\PY{o}{+}\PY{l+s+s1}{\PYZsq{}}\PY{l+s+s1}{: }\PY{l+s+si}{\PYZpc{}.7f}\PY{l+s+se}{\PYZbs{}n}\PY{l+s+s1}{   }\PY{l+s+s1}{\PYZsq{}}\PY{o}{\PYZpc{}}\PY{k}{alpha\PYZus{}b} + u\PYZsq{}\PYZbs{}u00B7\PYZsq{} + \PYZsq{}f: \PYZpc{}.7f\PYZsq{}\PYZpc{}f0\PYZus{}b)
\end{Verbatim}


    \begin{Verbatim}[commandchars=\\\{\}]
Inexact Line Search Methods:
 - Fletcher solution
   ·α: 4.5730216
   ·f: -4.4062606
 - Backtracking solution
   ·α: 1.0000000
   ·f: 19.8410511

    \end{Verbatim}

    O gráfico a seguir mostra que a solução para o \(\alpha\) encontrado
pelo inexact line search do Fletcher está mais próxima do mínimo global.
No caso do Backtracking, como a condição de entrada do \emph{while} o
\(\alpha\) permaneceu igual a \(1.0\), como a direção inicial
\(\mathbf{d}_0\) aponta para o mínimo global, o resultado encontrado
pelo \(\alpha=1.0\) se mostrou pior que no item d).

    \begin{Verbatim}[commandchars=\\\{\}]
{\color{incolor}In [{\color{incolor}22}]:} \PY{k+kn}{from} \PY{n+nn}{run\PYZus{}exercises} \PY{k}{import} \PY{n}{plot\PYZus{}func\PYZus{}alpha}
         \PY{n}{plot\PYZus{}contour}\PY{p}{(}\PY{n}{x\PYZus{}0} \PY{o}{+} \PY{n}{alpha\PYZus{}f}\PY{o}{*}\PY{n}{d\PYZus{}0}\PY{p}{,} \PY{n}{x\PYZus{}0} \PY{o}{+} \PY{n}{alpha\PYZus{}b}\PY{o}{*}\PY{n}{d\PYZus{}0}\PY{p}{)}
\end{Verbatim}


    \begin{center}
    \adjustimage{max size={0.9\linewidth}{0.9\paperheight}}{output_53_0.png}
    \end{center}
    { \hspace*{\fill} \\}
    
    Novamente é possível ver que o a solução da busca em linha de Fletcher
encotnrou um \(\alpha\) mínimo da função
\(f(\mathbf{x}_0 + \alpha \mathbf{d}_0)\)

    \begin{Verbatim}[commandchars=\\\{\}]
{\color{incolor}In [{\color{incolor}23}]:} \PY{n}{alphas} \PY{o}{=} \PY{n}{np}\PY{o}{.}\PY{n}{linspace}\PY{p}{(}\PY{l+m+mi}{0}\PY{p}{,} \PY{l+m+mf}{5.7120}\PY{p}{)}
         \PY{n}{plot\PYZus{}func\PYZus{}alpha}\PY{p}{(}\PY{n}{x\PYZus{}0}\PY{p}{,} \PY{n}{d\PYZus{}0}\PY{p}{,} \PY{n}{alphas}\PY{p}{,} \PY{n}{alpha\PYZus{}f}\PY{p}{,} \PY{n}{f0\PYZus{}f}\PY{p}{,} \PY{n}{alpha\PYZus{}b}\PY{p}{,} \PY{n}{f0\PYZus{}b}\PY{p}{)}
\end{Verbatim}


    \begin{center}
    \adjustimage{max size={0.9\linewidth}{0.9\paperheight}}{output_55_0.png}
    \end{center}
    { \hspace*{\fill} \\}
    

    % Add a bibliography block to the postdoc
    
    
    
    \end{document}
